We should note that the ``hubs'' that connect the PCs to the routers were actually switches
instead of hubs when when we did the lab, so they were also learning. \\
When the ping command from PC1 to PC2 is executed, PC1 sends out an ARP request for PC2's IP
address. Router1 forwards the ARP request and maps PC1's hardware address to its FastEthernet0/0
interface. The switch (supposed to be hub) that connects Router1, Router2 and PC2 doesn't know
PC2 yet, so it forwards the ARP request to both PC2 and Router2. Router2 doesn't know PC2
either, so it forwards the request as well. The same goes for the switch that connects Router2,
Router3 and PC3. Indeed: in PC3's trace file we see the ARP request from PC1 asking for PC2's
hardware address.
\begin{lstlisting}
Frame 13: 60 bytes on wire (480 bits), 60 bytes captured (480 bits)
Ethernet II, Src: IntelCor_36:33:a0 (68:05:ca:36:33:a0), Dst: Broadcast (ff:ff:ff:ff:ff:ff)
    Destination: Broadcast (ff:ff:ff:ff:ff:ff)
    Source: IntelCor_36:33:a0 (68:05:ca:36:33:a0)
    Type: ARP (0x0806)
    Padding: 000000000000000000000000000000000000
Address Resolution Protocol (request)
    Hardware type: Ethernet (1)
    Protocol type: IP (0x0800)
    Hardware size: 6
    Protocol size: 4
    Opcode: request (1)
    Sender MAC address: IntelCor_36:33:a0 (68:05:ca:36:33:a0)
    Sender IP address: 10.0.1.11 (10.0.1.11)
    Target MAC address: 00:00:00_00:00:00 (00:00:00:00:00:00)
    Target IP address: 10.0.1.21 (10.0.1.21)
\end{lstlisting}
When PC2 responds to PC1's ARP request, Router1, as well as the switches in between, already
know about PC1, so they only forward the response to the right interfaces. \\
Upon learning PC2's hardware address, PC1 sends its ping message to PC2, and PC2 responds.
The ping succeeds. \\
Five seconds later we see an ARP request from PC2 asking for PC1's hardware address, even though no
ping message was ready to be sent at that point. We can't quite explain this request, as PC2
learned PC1's hardware address through PC1's ARP request. Perhaps it's an automatic refresh. \\
Then, PC2 pings PC1. Again, PC1 and PC2 are already known to all the
switches in between. Then, PC2 pings PC4. Router2 and Router3, as well as the switches in between,
don't know PC4 yet, so they all forward the request to all interfaces. Thus, even PC1 will receive
this request:
\begin{lstlisting}
Frame 16: 60 bytes on wire (480 bits), 60 bytes captured (480 bits)
Ethernet II, Src: IntelCor_36:31:f0 (68:05:ca:36:31:f0), Dst: Broadcast (ff:ff:ff:ff:ff:ff)
    Destination: Broadcast (ff:ff:ff:ff:ff:ff)
    Source: IntelCor_36:31:f0 (68:05:ca:36:31:f0)
    Type: ARP (0x0806)
    Padding: 000000000000000000000000000000000000
Address Resolution Protocol (request)
    Hardware type: Ethernet (1)
    Protocol type: IP (0x0800)
    Hardware size: 6
    Protocol size: 4
    Opcode: request (1)
    Sender MAC address: IntelCor_36:31:f0 (68:05:ca:36:31:f0)
    Sender IP address: 10.0.1.21 (10.0.1.21)
    Target MAC address: 00:00:00_00:00:00 (00:00:00:00:00:00)
    Target IP address: 10.0.1.41 (10.0.1.41)
\end{lstlisting}
Lastly, PC3 pings PC2. PC2 is already known to all bridges involved, so the ping request is only
forwarded by Router2. When PC3 sends its ARP request asking for PC2's hardware address (again,
we don't know why PC3 doesn't just derive this address from PC2's ping packet), all of the bridges
get to know PC3: we can find PC3's ARP request in the traces of all PCs. 