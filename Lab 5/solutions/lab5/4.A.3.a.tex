The three-way handshake packets are packets no. 1, 2 and 3 in ``Lab 5/traces/4.A.PC1.pcap''. \\
The first packet (from client (PC1) to server (PC2)) has the SYN flag set. PC2 recognizes this as the first packet of the three-way handshake. SYN is for synchronizing the sequence numbers. Say PC1 sent sequence number i, then PC2 knows that the next packet from PC1 will have sequence number i + 1. In this case, because the SYN flag is set, even though no application data was sent, the sequence number increases by 1. \\
The next packet (from server to client) has the SYN and ACK flags set. PC1 interprets this as the second packet in the three-way handshake. Say PC2 sent sequence number j, then PC1 knows the next packet from PC2 will have sequence number j + 1. This packet also acknowledges PC1's initial packet by sending acknowledgement number i + 1 (an ACK always has the sequence number of the first byte the host expects to receive next). \\
Then PC1 concludes the three-way handshake by sending one more packet with sequence number i + 1 and an acknowledgement of PC2's packet with sequence number j + 1.

\begin{lstlisting}
1	0.000000	10.0.5.11	10.0.5.22	TCP	74	48613→23 [SYN] Seq=2495102103 Win=29200 Len=0 MSS=1460 SACK_PERM=1 TSval=4215509 TSecr=0 WS=128
2	0.000621	10.0.5.22	10.0.5.11	TCP	74	23→48613 [SYN, ACK] Seq=3213249774 Ack=2495102104 Win=28960 Len=0 MSS=1460 SACK_PERM=1 TSval=4214383 TSecr=4215509 WS=128
3	0.000660	10.0.5.11	10.0.5.22	TCP	66	48613→23 [ACK] Seq=2495102104 Ack=3213249775 Win=29312 Len=0 TSval=4215509 TSecr=4214383
\end{lstlisting}