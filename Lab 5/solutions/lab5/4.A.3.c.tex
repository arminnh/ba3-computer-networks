Packet no. 4, the first packet after the three-way handshake, contains Telnet application data. The sequence number of the first byte from client to server is 2495102104, which is the same sequence number PC1 used in the final message of the three-way handshake. The relative sequence number is 1.

\begin{lstlisting}
Frame 4: 93 bytes on wire (744 bits), 93 bytes captured (744 bits)
Ethernet II, Src: IntelCor_1a:80:15 (68:05:ca:1a:80:15), Dst: IntelCor_1a:7c:75 (68:05:ca:1a:7c:75)
Internet Protocol Version 4, Src: 10.0.5.11 (10.0.5.11), Dst: 10.0.5.22 (10.0.5.22)
Transmission Control Protocol, Src Port: 48613 (48613), Dst Port: 23 (23), Seq: 2495102104, Ack: 3213249775, Len: 27
    Source Port: 48613 (48613)
    Destination Port: 23 (23)
    [Stream index: 0]
    [TCP Segment Len: 27]
    Sequence number: 2495102104
    [Next sequence number: 2495102131]
    Acknowledgment number: 3213249775
    Header Length: 32 bytes
    .... 0000 0001 1000 = Flags: 0x018 (PSH, ACK)
    Window size value: 229
    [Calculated window size: 29312]
    [Window size scaling factor: 128]
    Checksum: 0x1e62 [validation disabled]
    Urgent pointer: 0
    Options: (12 bytes), No-Operation (NOP), No-Operation (NOP), Timestamps
        No-Operation (NOP)
        No-Operation (NOP)
        Timestamps: TSval 4215509, TSecr 4214383
    [SEQ/ACK analysis]
Telnet
    Do Suppress Go Ahead
    Will Terminal Type
    Will Negotiate About Window Size
    Will Terminal Speed
    Will Remote Flow Control
    Will Linemode
    Will New Environment Option
    Do Status
    Will X Display Location
\end{lstlisting}

