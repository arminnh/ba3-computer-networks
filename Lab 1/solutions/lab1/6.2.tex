\begin{lstlisting}
student@lab2pc1:/$ ping -c 5 10.0.1.11
PING 10.0.1.11 (10.0.1.11) 56(84) bytes of data.
64 bytes from 10.0.1.11: icmp_seq=1 ttl=64 time=0.032 ms
64 bytes from 10.0.1.11: icmp_seq=2 ttl=64 time=0.017 ms
64 bytes from 10.0.1.11: icmp_seq=3 ttl=64 time=0.025 ms
64 bytes from 10.0.1.11: icmp_seq=4 ttl=64 time=0.030 ms
64 bytes from 10.0.1.11: icmp_seq=5 ttl=64 time=0.036 ms

--- 10.0.1.11 ping statistics ---
5 packets transmitted, 5 received, 0% packet loss, time 3996ms
rtt min/avg/max/mdev = 0.017/0.028/0.036/0.006 ms
\end{lstlisting}

We expected the pings to 10.0.1.11 to go through the network (eth0), unlike the pings to 127.0.0.1, but we noticed they both went through the loopback interface.\\ 
We found this on the web:\begin{verbatim}
"The loopback interface does not represent any actual hardware, but exists
so applications running on your computer can always connect to servers on
the same machine."
\end{verbatim}
The local Ethernet interface is used for packets that are meant for other hosts.\\
The loopback interface is used by a computer to communicate to itself.