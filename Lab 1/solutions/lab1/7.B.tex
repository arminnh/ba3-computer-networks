Issue a ping to the non-existing IP address 111.111.111.111:\\
   \begin{lstlisting}  
    connect: Network is unreachable
\end{lstlisting}  
Issue a ping to the broadcast address 10.0.1.255 using the command:
\begin{lstlisting}  
    14:58:25.587269 IP 10.0.1.11 > 10.0.1.255: ICMP echo request, id 9803, seq 1, length 64
    14:58:26.588257 IP 10.0.1.11 > 10.0.1.255: ICMP echo request, id 9803, seq 2, length 64
\end{lstlisting}  
    There was no reply to the requests because every host on the network had \\the \textit{net.ipv4.icmp\_echo\_ignore\_broadcasts} option enabled.\\
After disabling this with \textit{sysctl net.ipv4.icmp\_echo\_ignore\_broadcasts=0},\\ we get the following output:
\begin{lstlisting}	  
	14:59:27.536259 IP 10.0.1.11 > 10.0.1.255: ICMP echo request, id 9805, seq 1, length 64
	14:59:27.536820 IP 10.0.1.14 > 10.0.1.11: ICMP echo reply, id 9805, seq 1, length 64
	14:59:27.536850 IP 10.0.1.13 > 10.0.1.11: ICMP echo reply, id 9805, seq 1, length 64
	14:59:27.536860 IP 10.0.1.12 > 10.0.1.11: ICMP echo reply, id 9805, seq 1, length 64
	14:59:28.535238 IP 10.0.1.11 > 10.0.1.255: ICMP echo request, id 9805, seq 2, length 64
	14:59:28.535809 IP 10.0.1.14 > 10.0.1.11: ICMP echo reply, id 9805, seq 2, length 64
	14:59:28.535822 IP 10.0.1.13 > 10.0.1.11: ICMP echo reply, id 9805, seq 2, length 64
	14:59:28.535825 IP 10.0.1.12 > 10.0.1.11: ICMP echo reply, id 9805, seq 2, length 64
\end{lstlisting}

The three other hosts have now responded.
