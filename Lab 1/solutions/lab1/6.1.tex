From PC1, send 5 ping messages (using the -c option) to PC2. Save the output:
	student@lab2pc1:/$ ping -c 5 10.0.1.12
	PING 10.0.1.12 (10.0.1.12) 56(84) bytes of data.
	64 bytes from 10.0.1.12: icmp_seq=1 ttl=64 time=0.623 ms
	64 bytes from 10.0.1.12: icmp_seq=2 ttl=64 time=0.610 ms
	64 bytes from 10.0.1.12: icmp_seq=3 ttl=64 time=0.554 ms
	64 bytes from 10.0.1.12: icmp_seq=4 ttl=64 time=0.618 ms
	64 bytes from 10.0.1.12: icmp_seq=5 ttl=64 time=0.585 ms

	--- 10.0.1.12 ping statistics ---
	5 packets transmitted, 5 received, 0% packet loss, time 3997ms
	rtt min/avg/max/mdev = 0.554/0.598/0.623/0.025 ms

On PC2, issue a ping to the IP address of PC1:
student@lab2pc1:/$ ping -c 5 10.0.1.11
	PING 10.0.1.11 (10.0.1.11) 56(84) bytes of data.
	64 bytes from 10.0.1.11: icmp_seq=1 ttl=64 time=0.614 ms
	64 bytes from 10.0.1.11: icmp_seq=2 ttl=64 time=0.594 ms
	64 bytes from 10.0.1.11: icmp_seq=3 ttl=64 time=0.603 ms
	64 bytes from 10.0.1.11: icmp_seq=4 ttl=64 time=0.596 ms
	64 bytes from 10.0.1.11: icmp_seq=5 ttl=64 time=0.602 ms

	--- 10.0.1.11 ping statistics ---
	5 packets transmitted, 5 received, 0% packet loss, time 3999ms
	rtt min/avg/max/mdev = 0.594/0.601/0.614/0.031 ms

Also, issue a ping command to the loopback interface, 127.0.0.1:
	student@lab2pc1:/$ ping -c 5 127.0.0.1
	PING 127.0.0.1 (127.0.0.1) 56(84) bytes of data.
	64 bytes from 127.0.0.1: icmp_seq=1 ttl=64 time=0.032 ms
	64 bytes from 127.0.0.1: icmp_seq=2 ttl=64 time=0.031 ms
	64 bytes from 127.0.0.1: icmp_seq=3 ttl=64 time=0.024 ms
	64 bytes from 127.0.0.1: icmp_seq=4 ttl=64 time=0.032 ms
	64 bytes from 127.0.0.1: icmp_seq=5 ttl=64 time=0.030 ms

	--- 127.0.0.1 ping statistics ---
	5 packets transmitted, 5 received, 0% packet loss, time 3996ms
	rtt min/avg/max/mdev = 0.024/0.029/0.032/0.007 ms
