\newpage
PC1
\begin{lstlisting}
student@lab2pc1:~$ ping -c1 10.0.1.120
PING 10.0.1.120 (10.0.1.120) 56(84) bytes of data.
64 bytes from 10.0.1.120: icmp_seq=1 ttl=64 time=0.537 ms

--- 10.0.1.120 ping statistics ---
1 packets transmitted, 1 received, 0% packet loss, time 0ms
rtt min/avg/max/mdev = 0.537/0.537/0.537/0.000 ms



student@lab2pc1:~$ ping -c1 10.0.1.101
PING 10.0.1.101 (10.0.1.101) 56(84) bytes of data.
64 bytes from 10.0.1.101: icmp_seq=1 ttl=64 time=0.589 ms

--- 10.0.1.101 ping statistics ---
1 packets transmitted, 1 received, 0% packet loss, time 0ms
rtt min/avg/max/mdev = 0.589/0.589/0.589/0.000 ms



student@lab2pc1:~$ ping -c1 10.0.1.121
PING 10.0.1.121 (10.0.1.121) 56(84) bytes of data.
From 10.0.1.100 icmp_seq=1 Destination Host Unreachable

--- 10.0.1.121 ping statistics ---
1 packets transmitted, 0 received, +1 errors, 100% packet loss, time 0ms
\end{lstlisting}

PC4
\begin{lstlisting}
student@lab2pc1:~$ ping -c1 10.0.1.100
connect: Network is unreachable
\end{lstlisting}

PC2
\begin{lstlisting}
student@lab2pc1:~$ ping -c1 10.0.1.121
connect: Network is unreachable

student@lab2pc1:~$ ping -c1 10.0.1.120
connect: Network is unreachable
\end{lstlisting}


a and b succeed, all of the others fail. \\

PC1 and PC3 have the same netmask and are in the same subnet. PC1 successfully retrieves PC3's hardware address through an ARP request. \\
PC1 and PC2 don't have the same netmask, but PC1's subnet would match PC2's subnet if they both had netmask /24, so PC1 sends an ARP request to PC2. PC2 receives the ARP request and "coincidentally", they would also be in the same subnet if they both had netmask /28, so PC2 responds to the ARP request and the ping succeeds. \\
PC1 and PC4 don't have the same netmask, but they would be in the same subnet if they both had netmask /24, so PC1 sends an ARP request to PC4. PC4 receives the request, but from its point of view, PC1 is not in the same subnet, even if they both had netmask /28, so it doesn't respond to the ARP request and the ping fails. \\
Because of the reason PC4 didn't respond to PC1's ARP request, the ping fails right away with a "Network is unreachable" error. \\
PC2 and PC4 have the same netmask, but are not in the same subnet, so we get a "Network is unreachable" error again. \\
For the ping from PC2 to PC3, the situation is the same as with PC4 to PC1.
