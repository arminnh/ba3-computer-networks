PC1's routing table:
\begin{lstlisting}
default         10.0.1.21       0.0.0.0         UG        0 0          0 eth0
10.0.0.0        10.0.1.71       255.255.0.0     UG        0 0          0 eth0
10.0.1.0        *               255.255.255.0   U         0 0          0 eth0
10.0.3.0        10.0.1.61       255.255.255.0   UG        0 0          0 eth0
10.0.3.9        10.0.1.81       255.255.255.255 UGH       0 0          0 eth0
\end{lstlisting}
Number of matches:\\
10.0.3.9: 4\\
10.0.3.14: 3\\
10.0.4.1: 2\\

We see ARP messages asking for 10.0.1.81, followed by messages asking for 10.0.1.61, followed by messages asking for 10.0.1.71.\\

Longer network prefixes are always preferred. \\
That's why, for a message sent to 10.0.3.9, PC1 will match the table entry 10.0.3.9/32, thus explaining the ARP message asking for 10.0.1.81.\\
10.0.3.14 doesn't match any table entries of network prefix length 32, but it does match 10.0.3.0/24, thus explaining the ARP message asking for 10.0.1.61.\\
10.0.4.1 doesn't match any table entries of network prefix length 24, but it does match 10.0.0.0/16, thus explaining the ARP message asking for 10.0.1.71\\