Before:
\begin{lstlisting}
Router1#show ip route
Codes: C - connected, S - static, R - RIP, M - mobile, B - BGP
       D - EIGRP, EX - EIGRP external, O - OSPF, IA - OSPF inter area 
       N1 - OSPF NSSA external type 1, N2 - OSPF NSSA external type 2
       E1 - OSPF external type 1, E2 - OSPF external type 2
       i - IS-IS, su - IS-IS summary, L1 - IS-IS level-1, L2 - IS-IS level-2
       ia - IS-IS inter area, * - candidate default, U - per-user static route
       o - ODR, P - periodic downloaded static route

Gateway of last resort is not set

     10.0.0.0/24 is subnetted, 2 subnets
C       10.0.2.0 is directly connected, FastEthernet0/0
C       10.0.3.0 is directly connected, Vlan1
\end{lstlisting}

After:
\begin{lstlisting}
Router1#show ip route
Codes: C - connected, S - static, R - RIP, M - mobile, B - BGP
       D - EIGRP, EX - EIGRP external, O - OSPF, IA - OSPF inter area 
       N1 - OSPF NSSA external type 1, N2 - OSPF NSSA external type 2
       E1 - OSPF external type 1, E2 - OSPF external type 2
       i - IS-IS, su - IS-IS summary, L1 - IS-IS level-1, L2 - IS-IS level-2
       ia - IS-IS inter area, * - candidate default, U - per-user static route
       o - ODR, P - periodic downloaded static route

Gateway of last resort is not set

     10.0.0.0/24 is subnetted, 3 subnets
C       10.0.2.0 is directly connected, FastEthernet0/0
C       10.0.3.0 is directly connected, Vlan1
S       10.0.1.0 [1/0] via 10.0.2.22
\end{lstlisting}

The first column displays the flags that apply to each entry. The flags are explained above. For the entries in our tables, we see C and S. C means that the destination subnet is directly connected to the router. S means that the entry is static. We did indeed add the static entry ourselves.

How the routing table changes: An entry was added for subnet 10.0.1.0/24, for packets to be forwarded to 10.0.2.22 (PC2 eth1)