%!TEX root = labo.tex

\subsubsection*{Network Commands in Linux}
Read the manual pages of the following commands at \url{http://manpages.ubuntu.com/} for the operating system version ``\osversion'':
\begin{itemize}
	\item \cmd{route}
	\item \cmd{traceroute}
	\item \cmd{minicom}: This lab uses the \cmd{minicom} utility program to establish a serial connection between a Linux PC and a Cisco router.
\end{itemize}

\subsubsection*{{Proxy ARP}}
Go to the website of Cisco at \url{http://goo.gl/ixuktT} and read about Proxy ARP.

\subsubsection*{Cisco IOS}
The Cisco routers in the Lab are running a recent version of the Cisco Internet Operating System (IOS). Read about the IOS at \url{http://goo.gl/UD23vX}. Note that this is reference material that you can use. You are not expected to go through all of the manuals listed here!

\remark The most useful manuals for this course are the ``IP Application Services Configuration Guide'' and ``Cisco IOS IP Switching Configuration Guide''.

\newpage
\subsection*{Prelab Questions}

\begin{questions}
	\q{1}{What is the IOS command to change the MTU (maximum transmission unit) for an interface on a Cisco router?}
	\q{2}{How does a router determine whether a datagrams to particular host can be directly delivered through one of its interfaces?}
	\q{3}{Which systems generate ICMP Route Redirect messages? Routers, hosts, or both?}
	\q{4}{What is the default maximum TTL value used by traceroute when sending UDP datagrams?}
	\q{5}{Describe the role of a default gateway in a routing table?}
	\q{6}{What is the network prefix of IP address 192.110.50.3/24?}
	\q{7}{Explain the difference between an IP address and a network prefix.}
	\q{8}{An organization has been assigned the network number 140.25.0.0/16 and it needs to create networks that support up to 60 hosts on each IP network. What is the maximum number of networks that can be set up? Explain your answer.}
\end{questions}
