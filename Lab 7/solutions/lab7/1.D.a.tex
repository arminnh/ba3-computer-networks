When downloading a file from an FTP server, we see that a PORT command is sent by the client
before the transfer starts.
This is where the payload of FTP data carries information on IP addresses. \\ 
A client sends a PORT command to an FTP server to set up active mode. In active mode, an FTP
server will initiate a connection to the client, instead of waiting for a connection attempt
from the client. It can connect to the client, because the client has specified the address
and port number it is listening on in the PORT command. \\

Examples of PORT commands: packet no. 48 in ``/Lab 7/traces/1.D.PC3.pcap'', packet no. 29 and
79 in ``/Lab 7/traces/1.D.PC4.pcap'' \\

We note that packet 48 is sent by PC3 and is captured before going through Router2. Packet 79
is this same packet after going through Router2. \\
We can see that after the address translation, the address has also been changed in the FTP
payload. This is possible because the network address translator knows about the FTP protocol,
so it also knows where to find IP addresses in the payload. In addition of translating the
IP addresses in the IP header, it will also translate those found in the payload. This, of course,
won't work with any protocols that the translator doesn't know. \\
In our case, the FTP-DATA has destination 200.0.0.2:51441, this goes to Router2, and then is
sent to PC3.
