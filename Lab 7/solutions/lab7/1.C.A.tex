TELNET
(1) PC1% telnet 10.0.1.3		: SUCCESS
(2) PC1% telnet 128.143.136.1		: SUCCESS

(3) Router1# telnet 10.0.1.2		: SUCCESS
(4) Router1# telnet 128.143.136.1	: SUCCESS

(5) PC4% telnet 10.0.1.2		: UNREACHABLE

PINGS
(1) PC1% ping -c 3 10.0.1.3		: SUCCESS
(2) PC1% ping -c 3 128.143.136.1	: SUCCESS

(3) Router1# ping 10.0.1.2		: SUCCESS
(4) Router1# ping 128.143.136.1		: SUCCESS

(5) PC4% ping -c 3 10.0.1.2		: UNREACHABLE

For both, the following explanation holds:
\begin{enumerate}
\item In same private network, to local IP(Router1)
\item To public network, using the recently added NAT chain rule on PC2.
\item In same private network, to local IP(PC1)
\item To public network, using the recently added NAT chain rule on PC2.
\item Can't find this ip in the public network, connections with the private hosts with this private IP is done through NAT, and thus the public NAT IPs.
\end{enumerate}