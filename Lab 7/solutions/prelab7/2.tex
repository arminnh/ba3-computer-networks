IP was originally supposed to be an end-to-end protocol, where each IP address uniquely
identifies a host. Any host would then be able to address any other host using just its
IP address, which is the host-to-host communication the comment refers to. With NAT and
IP masquerading however, a host may be ``hidden'' behind a network address translator
(i.e. a router with NAT configured), along with other hosts behind the same router. As
such a host can't uniquely identify a host using an IP address anymore. More information
is needed, such as a port number. The NAT protocol changes the IP addresses when
resolving to the host in the local network. This way, the exact IP address is not used
in the full communication process and the NAT-router intervenes with IP datagrams.
